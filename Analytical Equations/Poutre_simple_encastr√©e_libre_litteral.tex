\documentclass[a4paper,10pt]{article}

\usepackage[utf8x]{inputenc}
\usepackage[T1]{fontenc}
\usepackage[french]{babel} 
\usepackage{lmodern} % Pour changer le pack de police
%\usepackage{layout}
\usepackage[top=3cm, bottom=3cm, left=3cm, right=3cm]{geometry}

%\usepackage{setspace}
%\usepackage{soul}
%\usepackage{ulem}
%\usepackage{eurosym}
%\usepackage{bookman}
%\usepackage{charter}
%\usepackage{newcent}
%\usepackage{lmodern}
%\usepackage{mathpazo}
%\usepackage{mathptmx}
%\usepackage{url}
%\usepackage{verbatim}
%\usepackage{moreverb}
%\usepackage{listings}
%\usepackage{fancyhdr}
%\usepackage{wrapfig}
%\usepackage{color}
%\usepackage{colortbl}
\usepackage{amsmath}
\usepackage{amssymb}
\usepackage{mathrsfs}
%\usepackage{asmthm}
%\usepackage{makeidx}
\author{\textsc{Nom} Prénom}
\date{\today} 
\title{Mon titre d'article}
\begin{document}
\begin{abstract}
Le résumé (abstract en anglais) de mon article.
\end{abstract}

\begin{equation} %debut math
 \frac{E I}{\rho A}  \frac{\partial^4 w(x,t)}{\partial x^4} + \frac{\partial^2 w(x,t)}{\partial t^2  } = 0
 \end{equation}

Résolution par la méthode de séparation des variables.\\
On pose : \begin{math} w(x,t) = X(x)T(t)\end{math}.                                                      
L'équation (1) devient 
\begin{equation} %debut math
 \mu ^2   \frac{X^{(4)}(x)}{X(x)} = - \frac{ \ddot T(t)}{T(t)}  = \omega^2  \text{\qquad avec\quad} \mu = \sqrt{ \frac{E I}{\rho A} }  \end{equation}
Ce qui permet de poser deux équations séparées. L'équation pour la fonction dépendant du temps.
\begin{equation} \ddot T(t)+ \omega^2 T(t) = 0  \end{equation}
Ce qui conduit à
\begin{equation} T(t) = a sin(\omega t) + b cos(\omega t) \end{equation}

Et pour la fonction dépendant de la position
\begin{equation} X^{(4)} - \beta^4 X(x) = 0 \text{\qquad avec\quad} \beta^4 = \frac{\omega^2}{\mu^2}=  \frac{\rho A \omega^2}{EI}    \end{equation}

Ce qui, si on suppose des solutions de la forme \begin{math} D e^{sx} \end{math}, avec \begin{math} s \text{ et }  d  \end{math}  des constantes à déterminer, donne 
\begin{equation} s^4 - \beta^4 = 0\end{equation}
soit 
\[  %begin of math
s_{1,2}^2 = \beta^2 \Rightarrow 
  \left\{
      \begin{aligned}
        s_1	&=&	\beta\\
        s_2	&=&	-\beta\\
      \end{aligned}
    \right.
\text{\qquad et\quad}
s_{3,4}^2 = -\beta^2 \Rightarrow 
  \left\{
      \begin{aligned}
        s_3	&=&	j\beta\\
        s_4	&=&	-j\beta\\
      \end{aligned}
    \right.
\] %end of math
D'où la solution 
\begin{equation} X(x) = D_1 e^{\beta x} +  D_2 e^{-\beta x} +  D_3 e^{j\beta x} +  D_4 e^{-j\beta x}\end{equation}
Ou, sous une forme alternative 
\begin{equation}  X(x) = C_1 sin{\beta x} +  C_2 cos{\beta x} +  C_3 sinh{\beta x} +  C_4 cosh{\beta x} \end{equation}

Utilisation des conditions aux limites pour déterminer les constantes \begin{math} D_1 \end{math} à \begin{math} D_4 \end{math} ou  \begin{math} C_1 \end{math} à \begin{math} C_4 \end{math}
\\
Pour la poutre de longueur \begin{math} L \end{math} encastrée en \begin{math} x=L \end{math} et libre en \begin{math} x=0 \end{math}, les conditions aux limites s'écrivent
\begin{equation} 
	\left\{
		\begin{aligned}
			X(L) 	&=& 	0  &\Rightarrow  D_1 e^{\beta L} +  D_2 e^{-\beta L} +  D_3 e^{j\beta L} +  D_4 e^{-j\beta L} =  0 \\
			X'(L) 	&=&	0  &\Rightarrow  \beta ( D_1 e^{\beta L} -  D_2 e^{-\beta L} + j D_3 e^{j\beta L} - j D_4 e^{-j\beta L} ) =  0\\
			X''(0)	&=&	0  &\Rightarrow  \beta^2 ( D_1  +  D_2 -   D_3  -  D_4 ) =  0  \\
			X^{(3)} &=& 0  &\Rightarrow  \beta^3 ( D_1  -  D_2 -  j D_3  + j D_4 ) =  0  \\
		\end{aligned}	
	\right.
 \end{equation}

Ces équations se mettent sous forme matricielle

\begin{equation} 
	\begin{pmatrix}
   	e^{\beta L} 	& e^{-\beta L} 	& e^{j\beta L} 	& e^{-j\beta L}	\\
   	 e^{\beta L} & -e^{-\beta L}	&j e^{j\beta L}	& -je^{-j\beta L}	\\
  	 \beta^2	&\beta^2		&-\beta^2		&-\beta^2		\\
	\beta^3	&-\beta^3		&-j\beta^3		&j\beta^3		\\
	\end{pmatrix} 
	\begin{pmatrix}
   	D_1 	\\
   	D_2	\\
  	D_3	\\
	D_4	\\
	\end{pmatrix} 
= 
	\begin{pmatrix}
   	0 	\\
   	0	\\
  	0	\\
	0	\\
	\end{pmatrix} 
	\Rightarrow
	A.D=0
\end{equation}

Ce système admet une solution non triviale, si et seulement si \begin{math} det(A)=0 \end{math}. Ce qui conduit à 
\begin{equation} det(A) = -2j\beta^6 e^{-\beta L(1+2j)} [e^{j\beta L} + e^{3j\beta L} + 4e^{\beta L(1+2j)} + e^{\beta L (2+j)} + e^{\beta L(2+3j)} ] = 0     \end{equation}
\begin{equation} \iff e^{\beta L(1+2j)} [ e^{\beta L (-1-j)} + e^{\beta L (-1+j)} + 4 + e^{\beta L(1-j)} + e^{\beta L(1+j)} ] = 0  \end{equation}
\begin{equation}\iff (e^{j\beta L} + e^{-j\beta L})*(e^{\beta L}+e^{-\beta L})+4 = 0  \end{equation}
Ce qui conduit à 
\begin{equation} cos\beta L * cosh \beta L = -1  \end{equation}


Pour l'autre forme de X(x), c'est à dire l'équation (8*), les conditions aux limites s'écrivent, pour une poutre encastrée en \begin{math} x=0 \end{math} et libre en \begin{math} x=L \end{math}



\begin{equation}
	\left\{
		\begin{aligned}
			X(L) 	&=& 	0  &\Rightarrow  C_2   +  C_4 =  0 \\
			X'(L) 	&=&	0  &\Rightarrow  C_1 + C_3  =  0\\
			X''(0)	&=&	0  &\Rightarrow  \beta^2 ( -C_1 sin(\beta L)  - C_2 cos(\beta L)  +  C_3 sinh(\beta L)  + C_4 cosh(\beta L) ) =  0  \\
			X^{(3)} &=& 0  &\Rightarrow  \beta^3 ( -C_1 cos(\beta L)  + C_2 sin(\beta L)  +  C_3 cosh(\beta L)  + C_4 sinh(\beta L) ) =  0  
		\end{aligned}
	\right.
\end{equation}


Ces équations se mettent sous forme matricielle

\begin{equation} 
	\begin{pmatrix}
   	0 	& 1	& 0 	& 1	\\
   	 1 & 0	& 1	& 0	\\
  	 -sin(\beta L)	&-cos(\beta L)		&sinh(\beta L)		&cosh(\beta L)		\\
	-cos(\beta L)	&sin(\beta L)		&cosh(\beta L)		&sinh(\beta L)		\\
	\end{pmatrix} 
	\begin{pmatrix}
   	C_1 	\\
   	C_2	\\
  	C_3	\\
	C_4	\\
	\end{pmatrix} 
= 
	\begin{pmatrix}
   	0 	\\
   	0	\\
  	0	\\
	0	\\
	\end{pmatrix} 
	\Rightarrow
	A.C=0
\end{equation}

De la même manière, ce système n'admet une solution non triviale, si et seulement si \begin{math} det(A)=0 \end{math}. Ce qui conduit à 
\begin{equation} det(A) =cosh^2 + 2cos.cosh - cos^2 - sin  h^2 + sin^2 = 0   \end{equation}
\begin{equation}\iff 2cos.cosh = (sinh^2 - cosh^2) - (cos^2 + sin^2)   \end{equation}
Ce qui conduit à 
\begin{equation} cos\beta L * cosh \beta L = -1  \end{equation}

Les 2 formes de \begin{math} X(x) = D_1 e^{\beta x} +  D_2 e^{-\beta x} +  D_3 e^{j\beta x} +  D_4 e^{-j\beta x} =  C_1 sin{\beta x} +  C_2 cos{\beta x} +  C_3 sinh{\beta x} +  C_4 cosh{\beta x} \end{math} conduisent bien à la même condition sur \begin{math} \beta_n \end{math}.
\\\\
Cette équation (19*) est vérifiée par une infinité de valeurs \begin{math} \beta_n \end{math} que l'on calcule numériquement. 

\begin{tabular}{|c|c|}
\hline
n & \begin{math} \beta_n \end{math} \\ \hline
1 & 1.87510407 \\ \hline
2 & 4.69409113 \\ \hline
3 & 7.85475744	\\ \hline
4 & 10.99554073	\\ \hline
5 & 14.13716839	\\ \hline
\end{tabular}
\\\\
Pour des n grand, \begin{math} n>5 \end{math}, le \begin{math} cosh \end{math} deveint très grand, et l'équation (14*) ne peut être vérifiée que si le \begin{math} cos \end{math} est très proche de 0. On utilise donc comme approximation \begin{equation} \beta_n = \frac{(2n-1)\pi}{2} \text{\qquad avec\quad} n>5\end{equation}

A partir de la quantification de de \begin{math} \beta_n \end{math} et de l'équation (5), on obtient les pulsations naturelles
\begin{equation}  \beta^4 = \frac{\omega^2}{\mu^2}\Rightarrow  \omega_n = (\beta_n L)^2 \sqrt{  \frac{EI}{\rho A L^4}}  \text{\qquad avec\quad} n=1,2,3,... \end{equation}

Pour obtenir les \textbf{déformées modales}, on utilise les relations entre les coefficients \begin{math} C_1 \; à \;  C_4 \end{math} du système (15*).
Ce qui conduit à \begin{math} C_1=-C_3 (15b*) \end{math} et \begin{math}  C_2 = -C_4 (15a*)\end{math}.
D'où

\begin{equation}(15c*) \iff C_3 sin(\beta_n L)  + C_4 cos(\beta_n L)  +  C_3 sinh(\beta_n L)  + C_4 cosh(\beta_n L)  =  0  \end{equation}
\begin{equation} \iff C_3 (sin + sinh) + C_4 (cos + cosh) = 0  \end{equation}
\begin{equation} \iff C3 = - \frac{cos + cosh}{sin + sinh}C_4 \end{equation}
\begin{equation} \iff \boxed{C3 = - \sigma_n C_4} \text{\qquad avec\quad} \sigma_n= \frac{cos(\beta_n L) + cosh(\beta_n L)}{sin(\beta_n L) + sinh(\beta_n L)}\end{equation}


La fonction de variable \begin{math} x  \end{math} peut alors s'écrire 

\begin{equation} 
	\begin{aligned}
			 X_n(x) 	&=	C_ n[ (cosh(\beta_n x) - cos(\beta_n x)) - \sigma_n (sinh(\beta_n x) - sin(\beta_n x))] \\
			  		&=	C_n \phi_n(x) \\
	\end{aligned}
\end{equation}s


En conclusion, l'équation du déplacement est (équation (4*) et (26*)
\begin{equation} \boxed{ w(x,t)= X(x) T(t) = \sum_{n=1}^{\infty} C_n \phi_n(x) [a\sin(\omega_n t) + b\cos(\omega_n t)] } \end{equation}







 




\end{document}
























